\documentclass{article}
\usepackage[whole]{bxcjkjatype}
\usepackage{amsmath}
\usepackage{amssymb}
\usepackage{amsthm}
\usepackage[all]{xy}
\usepackage{mathtools}

\begin{document}
\section*{問題}
$n \geq 1$ を整数とする.同じサイズの立方体型ランプを隙間なく $n^3$ 個並べて,全体として $n \times n \times n$ の大きな立方体を作る.すると,その外面には各面に$n \times n$ 個,計 $6n^2$ 個の小正方形(外側に見えているランプの外面)が現れる.各ランプは \texttt{on/off} のいずれかの状態をもち,外面の小正方形のうち一つを押すと,その小正方形とちょうど反対側にある小正方形を結ぶ直線上の $n$ 個のランプだけが同時に反転(on $\leftrightarrow$ off)する.全ランプの \texttt{on/off} の状態が与えられたとき,上記の操作の繰り返しのみですべてのランプを \texttt{off} にできるための必要十分条件をなるべく簡潔に記述し,その条件の正しさを証明せよ.


\section{Small example}
例として、まず \(2\times 2\times 2\) の立方体を考えます。点灯パターンは次のように書かれます:

\[x_1,x_2,x_3,x_4,x_5,x_6,x_7,x_8 \in \mathbb{2} \]

\[
\xymatrix{
    & x_5 \ar@{-}[ld] \ar@{-}[rr] & & x_6 \ar@{-}[ld] \ar@{-}[dd] \\
    x_1 \ar@{-}[rr] & & x_2 \ar@{-}[dd] \\
    & x_8 \ar@{-}[ld] \ar@{-}[uu] & & x_7 \ar@{-}[ld] \ar@{-}[ll] \\
    x_4 \ar@{-}[uu] & & x_3 \ar@{-}[ll]
}
\]
切り替えパターンは次のように書かれます:

\[y_1,y_2,y_3,y_4,y_5,y_6,y_7,y_8,y_9,y_{10},y_{11},y_{12}\]

\[
\xymatrix{
    & x_5 \ar@{-}[ld]|{y_1} \ar@{-}[rr]|{y_2} & & x_6 \ar@{-}[ld]|{y_3} \ar@{-}[dd]|{y_4} \\
    x_1 \ar@{-}[rr]|(0.3){y_5} & & x_2 \ar@{-}[dd]|(0.3){y_6} \\
    & x_8 \ar@{-}[ld]|{y_7} \ar@{-}[uu]|(0.3){y_8} & & x_7 \ar@{-}[ld]|{y_9} \ar@{-}[ll]|(0.3){y_{10}} \\
    x_4 \ar@{-}[uu]|{y_{11}} & & x_3 \ar@{-}[ll]|{y_{12}}
}
\]
1 回の切り替え操作は \(y_i = 1 (\mathrm{true})\) に対応します。切り替え後、光の結果は \(y_i\) と接続された \(x_n\) および \(x_m\) に対して
\(x_n\oplus y_i\)、\(x_m\oplus y_i\) となります。したがって、全てのライトが同一の点灯を持つというのは、与えられた \(x_1,\dots,x_8\) に対して、次の式を満たす \(y_1,\dots,y_{12}\) が存在する時です。
\[
(x_1\oplus y_1 \oplus y_5 \oplus y_{11}) 
\land \cdots \land 
(x_8\oplus y_8 \oplus y_7 \oplus y_{10})
\]
\begin{align}
&(x_{1} \oplus y_{1} \oplus y_{11} \oplus y_{5}) \land\\
&(x_{2} \oplus y_{3} \oplus y_{5} \oplus y_{6}) \land\\
&(x_{3} \oplus y_{12} \oplus y_{6} \oplus y_{9}) \land\\
&(x_{4} \oplus y_{11} \oplus y_{12} \oplus y_{7}) \land\\
&(x_{5} \oplus y_{1} \oplus y_{2} \oplus y_{8}) \land\\
&(x_{6} \oplus y_{2} \oplus y_{3} \oplus y_{4}) \land\\
&(x_{7} \oplus y_{4} \oplus y_{9} \oplus y_{10}) \land\\
&(x_{8} \oplus y_{10} \oplus y_{7} \oplus y_{8})
\end{align}
各辺に対して 2 回以上の切り替え操作の必要性を考慮する必要があります。どのタイミングでも、切り替え回数が奇数であれば 1 回の切り替えに帰着し、偶数であれば切り替えなしに帰着します。これは排他的論理和の計算によるものです。以上の議論から以下の連立方程式が立式できる。
\[
\begin{cases}
x_{1} \oplus y_{1} \oplus y_{11} \oplus y_{5} &= 1 \\
x_{2} \oplus y_{3} \oplus y_{5} \oplus y_{6} &= 1 \\
x_{3} \oplus y_{12} \oplus y_{6} \oplus y_{9} &= 1 \\
x_{4} \oplus y_{11} \oplus y_{12} \oplus y_{7} &= 1 \\
x_{5} \oplus y_{1} \oplus y_{2} \oplus y_{8} &= 1 \\
x_{6} \oplus y_{2} \oplus y_{3} \oplus y_{4} &= 1 \\
x_{7} \oplus y_{4} \oplus y_{9} \oplus y_{10} &= 1 \\
x_{8} \oplus y_{10} \oplus y_{7} \oplus y_{8} &= 1 
\end{cases}
\]
したがって
\[
\begin{cases}
y_{1} \oplus y_{11} \oplus y_{5} &= \neg x_{1} \\
y_{3} \oplus y_{5} \oplus y_{6} &= \neg x_{2} \\
y_{12} \oplus y_{6} \oplus y_{9} &= \neg x_{3} \\
y_{11} \oplus y_{12} \oplus y_{7} &= \neg x_{4} \\
y_{1} \oplus y_{2} \oplus y_{8} &= \neg x_{5} \\
y_{2} \oplus y_{3} \oplus y_{4} &= \neg x_{6} \\
y_{4} \oplus y_{9} \oplus y_{10} &= \neg x_{7} \\
y_{10} \oplus y_{7} \oplus y_{8} &= \neg x_{8} 
\end{cases}
\]
連立方程式を掃き出し法で解くことを考える。行基本変形によって矛盾行の存在の有無が解の存在に対応し、つまりボタン押下戦略の有無に対応する。係数が打ち消されるのは、排他的論理和によって打ち消しが発生したときのみであり、その可能性は全方程式をすべて足したときのみである。そこで、全両辺の排他的論理和をとってみると
\begin{align*}
(y_{1} \oplus y_{11} \oplus y_{5})\oplus\dots\oplus(y_{10} \oplus y_{7} \oplus y_{8})
&= \neg x_{1}\oplus\dots\oplus\neg x_{8} \\
0 &= \neg x_{1}\oplus\dots\oplus\neg x_{8} \\
0 &= x_{1}\oplus\dots\oplus x_{8}
\end{align*}
つまり \(y_i\)の値に関係なく \(x_j\)は上記の条件を満たす必要がある。そしてこのとき掃き出し法によって以下の解を得る。

\begin{align}
y_{1}  &= \neg x_{1}\oplus y_{5}\oplus y_{11}\\
y_{3}  &= \neg x_{2}\oplus y_{5}\oplus y_{6}\\
y_{12} &= \neg x_{3}\oplus y_{6}\oplus y_{9}\\
y_{7}  &= \neg x_{4}\oplus y_{11}\oplus y_{12}\\
y_{2}  &= \neg x_{5}\oplus y_{1}\oplus y_{8}\\
y_{4}  &= \neg x_{6}\oplus y_{2}\oplus y_{3}\\
y_{10} &= \neg x_{7}\oplus y_{4}\oplus y_{9}
\end{align}

\section{General case}

まず、ランプの点灯パターンを次のように表します:

\[
f: x_{i,j,k}\mapsto (y_{-,j,k},y_{i,-,k},y_{i,j,-})
\]

ここで、\(1 \leq i,j,k \leq n\) です。次に、切り替えパターンが
成立するための条件を次のように表します:

\[
\bigwedge_{\substack{1\le i\le n\\1\le j\le n\\1\le k\le n}}x_{i,j,k}\oplus y_{-,j,k}\oplus y_{i,-,k}\oplus y_{i,j,-}=1
\]

以上の議論から以下の連立方程式が立式できる

\[
\begin{cases}
    x_{1,1,1}\oplus y_{-,1,1}\oplus y_{1,-,1}\oplus y_{1,1,-} &= 1 \\
    &\vdots \\
    x_{i,j,k}\oplus y_{-,j,k}\oplus y_{i,-,k}\oplus y_{i,j,-} &= 1 \\
    &\vdots \\
    x_{n,n,n}\oplus y_{-,n,n}\oplus y_{n,-,n}\oplus y_{n,n,-} &= 1
\end{cases}
\]

すなわち、

\[
\begin{cases}
    y_{-,1,1}\oplus y_{1,-,1}\oplus y_{1,1,-} &= \lnot x_{1,1,1} \\
    &\vdots \\
    y_{-,j,k}\oplus y_{i,-,k}\oplus y_{i,j,-} &= \lnot x_{i,j,k} \\
    &\vdots \\
    y_{-,n,n}\oplus y_{n,-,n}\oplus y_{n,n,-} &= \lnot x_{n,n,n}
\end{cases}
\]

これが成立するのは、以下のとおりである

\[
\bigoplus_{\substack{y_=y_1,y_2\\y=y_1,y_2\\z=y_1,y_2}}\lnot x_{x,y,z}=1 \iff \bigoplus_{\substack{y_=y_1,y_2\\y=y_1,y_2\\z=y_1,y_2}} x_{x,y,z}=1
\]

したがって、この条件が必要十分条件となる。

\[\bigwedge_{\substack{1\le y_1 < y_2 \le n\\1\le y_1 < y_2 \le n\\1\le y_1 < y_2 \le n}} \bigoplus_{\substack{y_=y_1,y_2\\y=y_1,y_2\\z=y_1,y_2}}x_{x,y,z}=1\]


\end{document}
