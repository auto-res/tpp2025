\documentclass{article}

\usepackage{amsmath}
\begin{document}
\section*{問題}
$n \geq 1$ を整数とする.同じサイズの立方体型ランプを隙間なく $n^3$ 個並べて,
全体として $n \times n \times n$ の大きな立方体を作る.すると,その外面には各面に
$n \times n$ 個,計 $6n^2$ 個の小正方形(外側に見えているランプの外面)が現れる.
各ランプは \texttt{on/off} のいずれかの状態をもち,外面の小正方形のうち一つを押すと,
その小正方形とちょうど反対側にある小正方形を結ぶ直線上
の $n$ 個のランプだけが同時に反転(on $\leftrightarrow$ off)する.
全ランプの \texttt{on/off} の状態が与えられたとき,
上記の操作の繰り返しのみですべてのランプを \texttt{off} にできるための
必要十分条件をなるべく簡潔に記述し,その条件の正しさを証明せよ.

\section{Small example}
例として、まず \(2\times 2\times 2\) の立方体を考えます。
点灯パターンは次のように書かれます:
\[x_1,x_2,x_3,x_4,x_5,x_6,x_7,x_8 \in \mathbb{2} \]
\[
\xymatrix{
    & x_5 \ar@{-}[ld] \ar@{-}[rr] & & x_6 \ar@{-}[ld] \ar@{-}[dd] \\
    x_1 \ar@{-}[rr] & & x_2 \ar@{-}[dd] \\
    & x_8 \ar@{-}[ld] \ar@{-}[uu] & & x_7 \ar@{-}[ld] \ar@{-}[ll] \\
    x_4 \ar@{-}[uu] & & x_3 \ar@{-}[ll]
}
\]
切り替えパターンは次のように書かれます:
\[y_1,y_2,y_3,y_4,y_5,y_6,y_7,y_8,y_9,y_{10},y_{11},y_{12}\]
\[
\xymatrix{
    & x_5 \ar@{-}[ld]|{y_1} \ar@{-}[rr]|{y_2} & & x_6 \ar@{-}[ld]|{y_3} \ar@{-}[dd]|{y_4} \\
    x_1 \ar@{-}[rr]|(0.3){y_5} & & x_2 \ar@{-}[dd]|(0.3){y_6} \\
    & x_8 \ar@{-}[ld]|{y_7} \ar@{-}[uu]|(0.3){y_8} & & x_7 \ar@{-}[ld]|{y_9} \ar@{-}[ll]|(0.3){y_{10}} \\
    x_4 \ar@{-}[uu]|{y_{11}} & & x_3 \ar@{-}[ll]|{y_{12}}
}
\]
1回の切り替え操作は \(y_i = 1 (\mathrm{true})\) に対応します。  
切り替え後、光の結果は \(y_i\) と接続された \(x_n\) および \(x_m\) に対して  
\(x_n\oplus y_i\)、\(x_m\oplus y_i\) となります。したがって、全てのライトが同一の点灯を持つというのは、与えられた \(x_1,\dots,x_8\) に対して、次の式を満たす \(y_1,\dots,y_{12}\) が存在する時です。
\[
(x_1\oplus y_1 \oplus y_5 \oplus y_{11}) 
\land \cdots \land 
(x_8\oplus y_8 \oplus y_7 \oplus y_{10})
\]

\begin{align}
&(x_{1} \oplus y_{1} \oplus y_{11} \oplus y_{5}) \land\\
&(x_{2} \oplus y_{3} \oplus y_{5} \oplus y_{6}) \land\\
&(x_{3} \oplus y_{12} \oplus y_{6} \oplus y_{9}) \land\\
&(x_{4} \oplus y_{11} \oplus y_{12} \oplus y_{7}) \land\\
&(x_{5} \oplus y_{1} \oplus y_{2} \oplus y_{8}) \land\\
&(x_{6} \oplus y_{2} \oplus y_{3} \oplus y_{4}) \land\\
&(x_{7} \oplus y_{4} \oplus y_{9} \oplus y_{10}) \land\\
&(x_{8} \oplus y_{10} \oplus y_{7} \oplus y_{8})
\end{align}

各辺に対して2回以上の切り替え操作の必要性を考慮する必要があります。  
どのタイミングでも、切り替え回数が奇数であれば1回の切り替えに帰着し、偶数であれば切り替えなしに帰着します。  
これは排他的論理和の計算によるものです。
以上の議論から以下の連立方程式が立式できる。
\begin{displaymath}
\begin{cases}
x_{1} \oplus y_{1} \oplus y_{11} \oplus y_{5} &= 1 \\
x_{2} \oplus y_{3} \oplus y_{5} \oplus y_{6} &= 1 \\
x_{3} \oplus y_{12} \oplus y_{6} \oplus y_{9} &= 1 \\
x_{4} \oplus y_{11} \oplus y_{12} \oplus y_{7} &= 1 \\
x_{5} \oplus y_{1} \oplus y_{2} \oplus y_{8} &= 1 \\
x_{6} \oplus y_{2} \oplus y_{3} \oplus y_{4} &= 1 \\
x_{7} \oplus y_{4} \oplus y_{9} \oplus y_{10} &= 1 \\
x_{8} \oplus y_{10} \oplus y_{7} \oplus y_{8} &= 1 
\end{cases}
\end{displaymath}
したがって
\begin{displaymath}
\begin{cases}
y_{1} \oplus y_{11} \oplus y_{5} &= \neg x_{1} \\
y_{3} \oplus y_{5} \oplus y_{6} &= \neg x_{2} \\
y_{12} \oplus y_{6} \oplus y_{9} &= \neg x_{3} \\
y_{11} \oplus y_{12} \oplus y_{7} &= \neg x_{4} \\
y_{1} \oplus y_{2} \oplus y_{8} &= \neg x_{5} \\
y_{2} \oplus y_{3} \oplus y_{4} &= \neg x_{6} \\
y_{4} \oplus y_{9} \oplus y_{10} &= \neg x_{7} \\
y_{10} \oplus y_{7} \oplus y_{8} &= \neg x_{8} 
\end{cases}
\end{displaymath}
連立方程式を掃き出し法で解くことを考える。
行基本変形によって矛盾行の存在の有無が解の存在に対応し、つまりボタン押下戦略の有無に対応する。
係数が打ち消されるのは、排他的論理和によって打ち消しが発生したときのみであり、その可能性は
全方程式をすべて足したときのみである。そこで、全両辺の排他的論理和をとってみると
\begin{align*}
(y_{1} \oplus y_{11} \oplus y_{5})\oplus\dots\oplus(y_{10} \oplus y_{7} \oplus y_{8})
&= \neg x_{1}\oplus\dots\oplus\neg x_{8} \\
0 &= \neg x_{1}\oplus\dots\oplus\neg x_{8} \\
0 &= x_{1}\oplus\dots\oplus x_{8}
\end{align*}
つまり \(y_i\)の値に関係なく \(x_j\)は上記の条件を満たす必要がある。
そしてこのとき掃き出し法によって以下の解を得る。
\begin{align}
y_{1}  &= \neg x_{1}\oplus y_{5}\oplus y_{11}\\
y_{3}  &= \neg x_{2}\oplus y_{5}\oplus y_{6}\\
y_{12} &= \neg x_{3}\oplus y_{6}\oplus y_{9}\\
y_{7}  &= \neg x_{4}\oplus y_{11}\oplus y_{12}\\
y_{2}  &= \neg x_{5}\oplus y_{1}\oplus y_{8}\\
y_{4}  &= \neg x_{6}\oplus y_{2}\oplus y_{3}\\
y_{10} &= \neg x_{7}\oplus y_{4}\oplus y_{9}
\end{align}

\section{Larger example}
\[
\xymatrix{
    & & x_{1} \ar@{-}[ddd]|(0.33)\hole|(0.66)\hole \ar@{-}[rr] \ar@{-}[dl] & & x_{2} \ar@{-}[ddd]|(0.33)\hole|(0.66)\hole \ar@{-}[rr] \ar@{-}[dl] & & x_{3} \ar@{-}[ddd] \ar@{-}[dl]  \\
    & x_{4} \ar@{-}[ddd]|(0.33)\hole \ar@{-}[rr] \ar@{-}[dl] & & x_{5} \ar@{-}[ddd]|(0.33)\hole \ar@{-}[rr] \ar@{-}[dl] & & x_{6} \ar@{-}[ddd] \ar@{-}[dl] \\
    x_{7} \ar@{-}[ddd] \ar@{-}[rr] & & x_{8} \ar@{-}[ddd]|(0.33)\hole \ar@{-}[rr] & & x_{9} \ar@{-}[ddd]|(0.33)\hole \\

    & & x_{10} \ar@{-}[ddd]|(0.66)\hole \ar@{-}[rr]|(0.5)\hole \ar@{-}[dl] & & x_{11} \ar@{-}[ddd]|(0.66)\hole \ar@{-}[rr]|(0.5)\hole \ar@{-}[dl] & & x_{12} \ar@{-}[ddd] \ar@{-}[dl]  \\
    & x_{13} \ar@{-}[ddd]|(0.33)\hole \ar@{-}[rr]|(0.5)\hole \ar@{-}[dl] & & x_{14} \ar@{-}[ddd]|(0.33)\hole \ar@{-}[rr]|(0.5)\hole \ar@{-}[dl] & & x_{15} \ar@{-}[ddd] \ar@{-}[dl] \\
    x_{16} \ar@{-}[ddd] \ar@{-}[rr] & & x_{17} \ar@{-}[ddd]|(0.33)\hole \ar@{-}[rr] & & x_{18} \ar@{-}[ddd]|(0.33)\hole \\
    
    & & x_{19} \ar@{-}[rr]|(0.5)\hole \ar@{-}[dl] & & x_{20} \ar@{-}[rr]|(0.5)\hole \ar@{-}[dl] & & x_{21} \ar@{-}[dl]  \\
    & x_{22} \ar@{-}[rr]|(0.5)\hole \ar@{-}[dl] & & x_{23} \ar@{-}[rr]|(0.5)\hole \ar@{-}[dl] & & x_{24} \ar@{-}[dl] \\
    x_{25} \ar@{-}[rr] & & x_{26} \ar@{-}[rr] & & x_{27}
}
\]
\[(x_1\otimes y_{1,4,7}\otimes y_{1,2,3}\otimes y_{1,10,19})\land\dots\land(x_{27}\otimes y_{25,26,27}\otimes y_{21,24,27}\otimes y_{9,18,27})\]
\[y_{1,2,3}\otimes\dots\otimes y_{25,26,27}=x_1\otimes\dots\otimes x_{27}\]
\section{General case}
\[
f: x_{i,j,k}\mapsto (y_{-,j,k},y_{i,-,k},y_{i,j,-})
\]
\[
\bigwedge_{i=1,\dots,n}\bigwedge_{j=1,\dots,n}\bigwedge_{k=1,\dots,n}x_{i,j,k}\oplus y_{-,j,k}\oplus y_{i,-,k}\oplus y_{i,j,-}
\]
\[
\begin{cases}
    x_{1,1,1}\oplus y_{-,1,1}\oplus y_{1,-,1}\oplus y_{1,1,-} &= 1 \\
    &\vdots \\
    x_{i,j,k}\oplus y_{-,j,k}\oplus y_{i,-,k}\oplus y_{i,j,-} &= 1 \\
    &\vdots \\
    x_{n,n,n}\oplus y_{-,n,n}\oplus y_{n,-,n}\oplus y_{n,n,-} &= 1
\end{cases}
\]
掃き出し法の最中で矛盾行が存在するとすれば、それはすべての行を足した時のみである。
なぜならば係数行列を足した時0となるのは、各変数について偶数回ずつ足す時のみだからである。
しかし、変数の表れ方から係数がすべて偶数になりうるのは、総和の時のみであり、
それもnが偶数の時しか発生しない。
\[
\bigoplus_n y_{-,1,1}\oplus\dots\oplus\bigoplus_n y_{n,n,-} = x_{1,1,1}\oplus\dots\oplus x_{n,n,n}
\]
\[\bigoplus_n y_{*,*,*} = \begin{cases}
0 & \text{if n is even} \\
y_{*,*,*} & \text{if n is odd}
\end{cases}\]
\subsection{Odd case}
\end{document}